\documentclass[a4paper,10pt]{article} 

\usepackage[utf8]{inputenc} 
%\usepackage[T1]{fontenc}

\usepackage{textcomp}           % Extra Symbole (Grad Celsius etc.)
\usepackage{amssymb,amsmath}    % Schöne Formeln (AMS = American Mathematical Society)
\usepackage{graphicx}           % Bilder und Seitenränder
\usepackage{subcaption}			% captions for subfigures
\usepackage{booktabs}           % Schönere Tabellen
\usepackage{colortbl}           % Farbige Tabellen

%\usepackage{tcolorbox}			% schöne bunte Boxen
\usepackage{mathtools}			% \mathclap für ordentliche \underbrace-			environments
\usepackage{geometry}			% Pagelayout mit \newgeometry, \restoregeometry
\usepackage{float}
\usepackage{wrapfig}
\usepackage{enumitem}
\usepackage{float}
\usepackage{braket}
\usepackage{caption}

\graphicspath{{./img/}}


\bibliographystyle{unsrtnat}

\renewcommand{\k}{\mathbf{k}}
\begin{document}
\begin{titlepage}
 \begin{center}
	\Large{Advanced laboratory class 3}
	\end{center}
	\begin{center}
	 \LARGE{\textbf{FP3 - SQUID}}
	\end{center}
	
	\begin{center}
	
	\large Marco \textsc{Canteri} \\
	marco.canteri@student.uibk.ac.at
	\end{center}
	
	\begin{center}
	\vspace{1cm}
	Innsbruck, \today
	\vspace{2cm}
	\end{center}
	
	\begin{center}
	\includegraphics[scale=0.4]{img/uibk} 
	\end{center}

\end{titlepage}
\begin{abstract}

\end{abstract}
\section{Introduction}
\section{Theory}
\section{Experiment setup}

\section{Analysis}

 \begin{thebibliography}{99}


\bibitem{skriptum}
Fortgeschrittenenpraktikum 2, \textit{Entanglement and Bell’s inequality}. \textsc{Gregor Weihs, Kaisa Laiho, Harishankar Jayakumar}. WS 2015/16
\end{thebibliography}
\end{document}
