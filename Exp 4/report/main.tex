\documentclass[a4paper,10pt]{article} 

\usepackage[utf8]{inputenc} 
%\usepackage[T1]{fontenc}

\usepackage{textcomp}           % Extra Symbole (Grad Celsius etc.)
\usepackage{amssymb,amsmath}    % Schöne Formeln (AMS = American Mathematical Society)
\usepackage{graphicx}           % Bilder und Seitenränder
\usepackage{subcaption}			% captions for subfigures
\usepackage{booktabs}           % Schönere Tabellen
\usepackage{colortbl}           % Farbige Tabellen

%\usepackage{tcolorbox}			% schöne bunte Boxen
\usepackage{mathtools}			% \mathclap für ordentliche \underbrace-			environments
\usepackage{geometry}			% Pagelayout mit \newgeometry, \restoregeometry
\usepackage{float}
\usepackage{wrapfig}
\usepackage{enumitem}
\usepackage{float}
\usepackage{braket}
\usepackage{caption}

\graphicspath{{./img/}}


\bibliographystyle{unsrtnat}

\renewcommand{\k}{\mathbf{k}}
\begin{document}
\begin{titlepage}
 \begin{center}
	\Large{Advanced laboratory class 3}
	\end{center}
	\begin{center}
	 \LARGE{\textbf{FP3 - SQUID}}
	\end{center}
	
	\begin{center}
	
	\large Marco \textsc{Canteri} \\
	marco.canteri@student.uibk.ac.at
	\end{center}
	
	\begin{center}
	\vspace{1cm}
	Innsbruck, \today
	\vspace{2cm}
	\end{center}
	
	\begin{center}
	\includegraphics[scale=0.4]{img/uibk} 
	\end{center}

\end{titlepage}
\begin{abstract}

\end{abstract}
\section{Introduction}
\section{Theory}
\subsection{Superconductivity}
Superconductivity is the property of a medium to conduct current without any resistance. It was first discovered in 1911, when Heike Onnes cooled down mercury below 4.2 K \cite{firstsuperconductor}, finding out that the resistance dropped to zero. Since this discovery, superconductivity has been discovered in many elements and media which resistance drops to zero below a particular critical temperature $T_c$. There are different ways to classify superconductors, the easiest one is to divide superconductors in low-$T_c$ and high-$T_c$ superconductors. Low temperature superconductors are material with a critical temperature below $30$ K, while high temperature superconductors have $T_c$ greater than 30 K.\\
Superconductors can lose their property of superconductivity not only with a change in temperature, but also with a change of magnetic field or current, this leads to a different classification of superconductors: type I and type II superconductors. The former has a critical external magnetic field $H_c$ and behaves as superconductor below this field. Type II superconductors have two different critical fields, and they behave differently in various regimes.\\
In this experiment we used a Yttrium barium copper oxide (YBCO) superconductor which is a high-$T_c$ superconductors, historically is also the first high-$T_c$ superconductors found with a critical temperature above 77 K, the boiling point of liquid nitrogen. Indeed the critical temperature of YBCO is around 90-94 K, depending of the compound and purity.\\
The theoretical description of superconductivity can be done in different ways, the first classical approach is the simplest description using London equations, a quantum mechanical formulation is the Ginzburg-Landau (GL) theory which describes the macroscopic properties of superconductors. However this theory is still phenomenological, the first microscopic model is the BCS theory which was able to describe superconductivity from a microscopic point of view and has the GL theory as a limit. The key idea of BCS theory is that at low temperatures electrons pair up and form Copper pairs, creating a bosonic condensate. This condensate must be described as a single entity with a wavefunction $\psi(\mathbf{r},\theta) = |\psi(\mathbf{r})| e^{i\theta}$, which is not normalized, but $|\psi(\mathbf{r},\theta)|^2 = n_e$, where $n_e$ is the number of electrons in the superconductive state. These theory predicts the behaviour of low temperature superconductors, but fails to describe high temperature superconductors, whose description is still an open problem.


\subsection{Josephson effect and SQUID}

\section{Experiment setup}
\subsection{Shapiro steps and $e/h$}
\subsection{Critical temperature measurement}
\section{Analysis}
\subsection{Shapiro steps and $e/h$}
\subsection{Critical temperature measurement}
 \begin{thebibliography}{99}
\bibitem{firstsuperconductor} \textsc{H. K. Onnes} \textit{The resistance of pure mercury at helium temperatures}, Commun. Phys. Lab. Univ. Leiden, Vol. 12 (1911), 120 

\bibitem{skriptum} Fortgeschrittenenpraktikum 2, \textit{Entanglement and Bell’s inequality}. \textsc{Gregor Weihs, Kaisa Laiho, Harishankar Jayakumar}. WS 2015/16
\end{thebibliography}
\end{document}
