\documentclass[a4paper,10pt]{article}

\usepackage[utf8]{inputenc}
%\usepackage[T1]{fontenc}

\usepackage{textcomp}           % Extra Symbole (Grad Celsius etc.)
\usepackage{amssymb,amsmath}    % Schöne Formeln (AMS = American Mathematical Society)
\usepackage{graphicx}           % Bilder und Seitenränder
\usepackage{subcaption}			% captions for subfigures
\usepackage{booktabs}           % Schönere Tabellen
\usepackage{colortbl}           % Farbige Tabellen

%\usepackage{tcolorbox}			% schöne bunte Boxen
\usepackage{mathtools}			% \mathclap für ordentliche \underbrace-			environments
\usepackage[left=2cm,right=2cm,top=2cm,bottom=2cm]{geometry}			% Pagelayout mit \newgeometry, \restoregeometry
\usepackage{float}
\usepackage{wrapfig}
\usepackage{enumitem}
\usepackage{float}
\usepackage{braket}
\usepackage{caption}
\usepackage[per-mode=reciprocal,output-decimal-marker={.},binary-units=true,separate-uncertainty=true]{siunitx}
\usepackage[breaklinks=true,colorlinks=true,linkcolor=blue,urlcolor=blue,citecolor=blue]{hyperref}
\usepackage{physics}
\usepackage{url}
\usepackage{subcaption}
\usepackage{calrsfs}
\DeclareMathAlphabet{\pazocal}{OMS}{zplm}{m}{n}
\usepackage{tikz}
\usetikzlibrary{decorations, positioning, intersections, calc, shapes,arrows, scopes}
\usepackage{pgfplots}
\usepackage{bodegraph}
\usepackage{circuitikz}
\usepackage{chemfig}
\usepackage{chemformula}
\usepackage[toc,page]{appendix}
\graphicspath{{./img/}}
\usepackage{verbatim}

\DeclareSIUnit\elementarycharge{e}

\newcommand{\dif}{\mathrm{d}}

\bibliographystyle{unsrtnat}

\renewcommand{\k}{\mathbf{k}}
\begin{document}
\begin{titlepage}
 \begin{center}
	\Large{Advanced laboratory course 3}
	\end{center}
	\begin{center}
	 \LARGE{\textbf{FP3 - Rare gas clusters}}
	\end{center}

	\begin{center}

	\large Marco \textsc{Canteri} \\
	marco.canteri@student.uibk.ac.at\\
	\large Maximilian \textsc{Münst} \\
	maximilian.muenst@student.uibk.ac.at
	\end{center}

	\begin{center}
	\vspace{1cm}
	Innsbruck, \today
	\vspace{1cm}
	\end{center}

	\begin{abstract}
	In this experiment Neon clusters were created using the supersonic expansion technique. The isotope population of the clusters was analyzed in addition to the magic numbers in the Neon cluster spectrum. Also, a look was taken at the appearance energy of \ch{Ne+} and \ch{Ne2+}. Finally we studied the	impact of air as pick-up gas on Neon cluster growth.
    \end{abstract}
    \vspace{1cm}

	\begin{center}
	\includegraphics[scale=0.56]{img/uibk}
	\end{center}

\end{titlepage}


\section{Introduction}
There are several reasons as to why research on clusters is important in modern day science. For one, cluster are the link between molecules and bulk matter, meaning that clusters show, depending on parameters like size, composition etc., properties typical for either molecules or solid matter. An example would be the transition from discrete energy levels in molecules to energy bands known from solid body physics. Furthermore, knowledge of clusters is fundamental in nanotechnology, where clusters are the building blocks used in the industry \cite{script}.

\section{Theoretical Background}
In this experiment one produces Neon clusters by means of expanding Neon gas. According to \cite{script}, Neon is a rare gas with an average mass of \SI{20.18}{\atomicmassunit}, with an isotope abundance of \ch{^{20}Ne}: $90.92\%$, \ch{^{21}Ne}: $0.26\%$ and \ch{^{22}Ne}: $8.82\%$.
%What's neon? And what's the meaning of a neon cluster? Can we reach god with it? Can we finally understand the meaning of life?\\
%These and more questions will be answered in this section. You will be surprised that the answer is $42$.

\subsection{Gas expansion}
There are different ways to produce clusters, however, usually those methods utilize 4 steps of cluster generation:
\begin{itemize}
	\item Vaporization (producing sufficiently cold particles in the gas phase)
	\item Nucleation (Condensation of gaseous atoms to form a cluster nucleus)
	\item Growth (adding particles to existing nuclei)
	\item Coalescence (Merging of small clusters)
\end{itemize}
To start the formation of clusters the thermal energy has to be smaller than the binding energy of the cluster. To reach sufficiently low temperatures, the gas is first stored at a low temperature while being at a very high pressure. The gas can escape from this high pressure environment via a nozzle into a vacuum chamber. The rapid expansion changes the velocity distribution of the gas from a Maxwell-Boltzmann distribution to a floating Maxwellian distribution, meaning that the distribution of velocity gets narrower and more symmetrical around the mean velocity. \\
As one is dealing with adiabatic expansion the enthalpy is conserved, meaning that one can write
\begin{equation}
	c_\mathrm{p} T_0 = c_\mathrm{p} T + \frac{m u^2}{2},
\end{equation}
where $c_\mathrm{p}$ is the heat capacity at constant pressure, $m$ is the mass of the atom, $u$ is the drift velocity of the gas while $T$ and $T_0$ are the temperatures at and before expansion \cite{script}. From here it is possible to derive a set of equations for the behaviour of the temperature $T$, the pressure $p$ and the density $\rho$.
\begin{equation}
	\begin{split}
	T &= T_0 \left[1 + \frac{\gamma - 1}{2} M^2 \right]^{-1} \\
	p &= p_0 \left[1 + \frac{\gamma - 1}{2} M^2 \right]^{\frac{1}{1 - \gamma}} \\
	\rho &= \rho_0 \left[1 + \frac{\gamma - 1}{2} M^2 \right]^{\frac{\gamma}{1-\gamma}} \\
	\end{split}
\end{equation}
Here, $\gamma$ represents the adiabatic exponent and $M = u/v_s$ is the Mach number. As one can imagine, all these parameters decrease drastically during the expansion.

\subsection{Cluster formation}
To generate a dimer, i.e. a two atom cluster, a three body collision is necessary in order to conserve energy and momentum.
\begin{equation}
	\ch{A} + \ch{A} + \ch{A} \rightarrow \ch{2 A} + \ch{A}
\end{equation}
As the cluster grows, condensation heat is accumulated in the cluster and can only be emitted by release of a monomer, i.e. a single Neon atom. \\
Growth can be described by classical gas to fluid phase transitions. The vapour pressure line divides said states as is shown in Fig. \ref{fig_formation}. However growth only occurs once supersaturation appears. This is due to the surface tension of the cluster. According to \cite{script} the critical radius for cluster growth is
\begin{equation}
	r^\ast = \frac{2 \sigma m}{k_\mathrm{B} T \rho} \frac{1}{\ln(\phi_\mathrm{k})},
\end{equation}
where $\sigma$ describes the surface tension, $T$ the temperature, $\rho$ the density and $\phi_\mathrm{k} = p_\mathrm{k} / p_\infty$ is the supersaturation condition. Clusters bigger than their critical radius will accumulate more atoms, while those below evaporate monomers until the condition is fulfilled again.
\begin{figure}
	\centering
	\includegraphics[width = 0.5 \textwidth]{formation.png}
	\caption{Phase diagram containing a growth line of a cluster. The image was taken from \cite{script}. }
	\label{fig_formation}
\end{figure}
Fig. \ref{fig_formation} shows the growth path of a cluster. The cluster state follows an adiabatic line from the gaseous into the fluid state. But only when the supersaturation condition is fulfilled, the cluster acquires an additional atom. The resulting condensation heat brings the cluster back to the equilibrium between liquid and gaseous state. \\
Empirically, the mean cluster size is determined as
\begin{equation}
	\bar{N} \propto \frac{p_0 D^{1.5}}{T_0^{2.4}}.
\end{equation}
Again, $T_0$ and $p_0$ describe the temperature and pressure in the stagnation chamber. $D = d / \tan \theta$ represents the reduced diameter of the nozzle, where $\theta$ is half the opening angle of the nozzle. As one can see, reducing the temperature in the stagnation chamber has the highest impact on the mean size of the clusters.

\section{Experimental Setup}
Basically, the setup in this experiment has to first produce and grow clusters of Neon atoms and in a second step ionize said clusters. Finally the ionized clusters have to be detected.

\subsection{Cluster source}
In this setup cluster nuclei are generated by means of supersonic expansion of the source gas. Neon gas is stored in a stagnation chamber at roughly \SI{13}{\bar} and low temperature. It is then released into a vacuum chamber where the gas adiabatically expands at supersonic speed. It is noteworthy that the gas velocity is approximately constant during the expansion, but the quickly decreasing speed of sound gradually increases the local Mach number. \\
Fig. \ref{fig_expansion} shows a sketch of the expansion cone generated by the Neon gas. As there is a low but finite background pressure in the expansion chamber, two shock zones develop: a barrel shock, which is formed symmetrically around the central axis of the expansion zone and a termination shock at the end of the expansion region, referred to as the Mach disk. Together, these shock fronts shield what is called the zone of silence from the background gas, which means that in the zone of silence there is hardly any collisions with the background gas to be expected. According to \cite{script} the position of the Mach disk can be calculated as
\begin{equation}
	\frac{x_\mathrm{M}}{d} = 0.67 \sqrt{\frac{p_0}{p_\mathrm{b}}},
\end{equation}
where $d$ is the diameter of the nozzle, $p_0$ and $p_\mathrm{b}$ represent the stagnation pressure and the background pressure respectively. In fig. \ref{fig_expansion} the Mach disk is depicted once for an undistorted case as a dash-dotted line and once as a solid line which is distorted by a skimmer. \\
The skimmer collects only the central component of the gas jet and guides it into the cluster chamber, where the clusters are formed by nucleation. It has to be positioned with care, as if it is placed too close to the nozzle, one will see turbulences in the expansion, while placing it behind the Mach disk results in considerable loss in intensity, as the collisions with the background gas are to be expected. The formed clusters are subsequently collided with an electron beam, such that the neutral clusters are positively ionized via electron ionization. The ions now can be analyzed with a mass spectrometer, for this experiment we used a linear quadrupole mass filter and a channeltron detector.%\\
%The electron beam is created by evaporation from a filament, the beam is focused with a set of electron lenses and filtered by energy with an hemispherical electron monochromator.

\begin{figure}[htp!]
	\centering
	\includegraphics[width = 0.6 \textwidth]{expansion.png}
	\caption{Sketch of the expansion cone of the Neon gas. The figure was taken from \cite{script}. }
	\label{fig_expansion}
\end{figure}

\subsection{Electron monochromator}
As the setup used in this experiment is also used for research, we could not install or set up any components ourselves. All that could be done was tune the voltages on several electrostatic lenses.
\\
A hairpin filament serves as electron source in this setup. A current of roughly \SI{2.35}{\ampere} is passed through the filament which leads to an emitted current of several \SI{10}{\micro \ampere}. As these electrons are emitted roughly isotropically in all directions with their kinetic energies following a Boltzmann distribution, a field generated by an anode is necessary to collect the electrons of the source chamber and guide them into an array of 3-element lenses, which collimate and focus the electron beam. A depiction of the setup is shown in fig. \ref{setup}.
\\
The focused beam of electrons is then guided towards a hemispherical energy selector. As fig. \ref{setup} indicates, this component consists of two concentric hemispherical surfaces which are kept at different direct voltages, which leads to a constant spherically symmetric field in the monochromator. This way an energy filter is created, as electrons which are too fast will collide with the outer surface, while electrons that are too slow will be absorbed by the inner element. Variation of these two voltages allows for a variation of the energy of the electron beam, while the energy resolution is maintained.
\\
The energy filter is followed by a second set of electrostatic lenses, which again collect the flux of electrons after the monochromator and focuses the beam into the collision chamber. Positive ions are created upon impact of an electron onto a cluster. These resulting ions are guided into the quadrupole mass filter, which filters the charge-mass-ratio. Electrons that do not hit a target are collected in a Faraday cup at the back of the collision chamber.

\begin{figure}[H]
	\centering
	\includegraphics[width = 1 \textwidth]{setup.png}
	\caption{Layout of the used experimental setup, taken from \cite{script}. Electrons emitted by the filament are collected and guided by a set of electrostatic lenses into the hemispherical energy monochromator. Electrons with the right kinetic energy can pass and are guided via a second set of lenses into the collision chamber. }
	\label{setup}
\end{figure}

\subsection{Quadrupole mass filter}
The mass of the charged cluster is measured with a combination of a linear quadrupole mass filter and a channeltron detector. The channeltron can detect only the intensity of incoming ions, but not the $m/z$-ratio. Therefore a quadrupole filter is used to select the ratio at which the intensity is of interest. In the experiment a script allows for automated scanning over a range of different masses. \\
Quadrupole filters are remarkably simple in terms of their structure. They consist of four metal rods of either cylindrical or hyperbolic shape that are placed in a square configuration \cite{ms_book}. The voltages applied to the rods have a AC and a DC component. Opposite rods are generally on the same voltage, while the rods next to a specified rod would have an opposite sign on the applied voltage. \\
If an ion is moving along in the direction of the rods it will be influenced by the electric field generated by the rods. Depending on the mass of the ion, the amplitude and frequency of the voltages and the radius of the configuration the ion will either pass through the quadrupole or collide with one of the rods and be filtered. \\
However, the resolution of a quadrupole cannot be indefinitely increased by tuning parameters, because ultimately the accuracy in the machining of the filter is the limiting factor. \\
Usually quadrupole filters operate at unit resolution, as is also the case in this experiment. Unit resolution means that the resolution inaccuracy is one unit mass (as in atomic mass unit) independent of the $m/z$-ratio at which the quadrupole is currently operating \cite{ms_book}.

\section{Analysis}
\subsection{Isotope population}\label{section:isotopes}
In the first part of the experiment, we took a look at the isotope distribution of neon clusters. We analyzed the mass spectrum from 25 up to around \SI{130}{\atomicmassunit \per \elementarycharge} at a pressure of \SI{1.3(1)e-4}{\milli \bar}. This should be enough to look for clusters with size up to \ch{Ne6+}. In fig. \ref{isotopespectrum} the full spectrum can be seen. The first peak appears at around 28. This peak is not due to Neon clusters, but stems from residual air in the setup. In fact air is mostly molecular nitrogen \ch{N_2} which has a weight of approximately $\SI{28}{\atomicmassunit}$. The other peaks are at 40, 60, 80, 100 and \SI{120}{\atomicmassunit} which are respectively \ch{Ne_2+}, \ch{Ne_3+}, \ch{Ne_4+}, \ch{Ne_5+}, and \ch{Ne_6+}. For the isotope population we can take at look at the peak around \SI{40}{\atomicmassunit}. We can see two peaks at 40.5 and 42.5. The neon abundance is 90.92\% for \ch{^{20}Ne}, 8.82\% for \ch{^{22}Ne}, and 0.26\% for \ch{^{21}Ne} \cite{script}, therefore we expect the first peak to correspond to a neon dimer of two \ch{^{20}Ne}, and the second peak corresponding to a neon dimer made of one \ch{^{20}Ne} and one \ch{^{22}Ne}. However the $m/z$ ratios do not match exactly the theoretical prediction of respectively 39.98 and \SI{41.98}{\atomicmassunit \per \elementarycharge} \cite{umc}. Apparently there is a systematic error of 0.5, maybe due to the calibration of the linear quadrupole mass filter. To further investigate this problem we can take a look at the measurement around \SI{60}{\atomicmassunit}, here we can see three peaks at 60.5, 62.5 and 64.5, these are due to \ch{Ne3+} first made of three \ch{^{20}Ne}, and then made of two \ch{^{20}Ne} and one \ch{^{22}Ne}, and one \ch{^{20}Ne} and two \ch{^{22}Ne} respectively. The calculated weight for this molecules are 59.98, 61.98, and 63.98, again we see the same pattern of systematic error. This is also confirmed by the two peaks around 80. We measured 80.6 and 82.6, which are for \ch{Ne4+} without isotopes and \ch{Ne4+} with one \ch{^{22}Ne} isotope, their theoretical values should be 79.97 and 81.97. For the cluster \ch{Ne5+} we have a peak at 100.7, another one at 102.6 and a small peak at 104.7, these are for \ch{Ne5+} without isotopes, \ch{Ne5+} with one \ch{^{22}Ne} isotope, and \ch{Ne5+} with two \ch{^{22}Ne} isotopes, their calculated values are 99.96, 101.96, and 103.96 respectively. For the last cluster \ch{Ne6+} we are able to see again three peaks at 120.7, 122.7, and 124.7 which correspond to \ch{Ne6+} without isotopes, \ch{Ne6+} with one \ch{^{22}Ne} isotope, and \ch{Ne6+} with two \ch{^{22}Ne} isotopes, their calculated values are 119.95, 121.95, and 123.95 respectively. In table \ref{isotopesresults} these results are summarized.  Furthermore we analyzed the height of the peaks to check whether it is proportional to the relative abundance of the isotopes as we expect. For instance we took the peaks around 40, the proportion between the two peaks is 4.16:1 , which is not exactly as the expected 4.9:1 ratio, nevertheless is pretty close and confirm our hypothesis on the isotopes. For all the other peaks we have the same situation. The slightly disagreement could be explained in several ways, maybe our source has a different abundance in isotopes or it could be that the probabilities for \ch{^{20}Ne} and \ch{^{22}Ne} to form a dimer or a bigger cluster are different due to slightly different structure, binding energy or other chemical properties.

\begin{figure}[H]
	\centering
	\includegraphics[width =\textwidth]{isotopespectrum}
	\caption{Recorded mass spectrum of Neon clusters.}
	\label{isotopespectrum}
\end{figure}

\begin{table}[H]
\centering
\caption{Results for the Neon cluster spectrum. All theoretical calculations are made with the software `Universal Mass Calculator' \cite{umc}. }\label{isotopesresults}
\begin{tabular}{ccc} \toprule
Measured peak (u/e) & Theoretical (u/e) & Isotope (charge +) \\ \midrule
40.5 & 39.98 & \ch{^{20}Ne2}\\
42.5 & 41.98 & \ch{(^{20}Ne)(^{22}Ne)}\\\midrule
60.5 & 59.98 & \ch{^{20}Ne3}\\
62.5 & 61.98 & \ch{(^{20}Ne)2(^{22}Ne)}\\
64.5 & 63.98 & \ch{(^{20}Ne)(^{22}Ne)2}\\\midrule
80.6& 79.97 & \ch{^{20}Ne4}\\
82.6& 81.97 & \ch{(^{20}Ne)3(^{22}Ne)}\\\midrule
100.7 & 99.96 & \ch{^{20}Ne5}\\
102.7 & 101.96 & \ch{(^{20}Ne)4(^{22}Ne)}\\
104.7 & 103.96 & \ch{(^{20}Ne)3(^{22}Ne)2}\\\midrule
120.7 & 119.95 & \ch{^{20}Ne6}\\
122.7 & 121.95 & \ch{(^{20}Ne)5(^{22}Ne)}\\
124.7 & 123.95 & \ch{(^{20}Ne)4(^{22}Ne)2}\\
\bottomrule
\end{tabular}
\end{table}


\subsection{Magic numbers}
In the next part of the experiment a measurement was conducted to look at magic numbers in the mass spectrum. Analogous to nuclear physics, magic numbers are understood to be numbers of atoms in a cluster where the cluster shows extraordinary stability, i.e. there is a sudden change in the amplitude of a peak or the slope between the peaks changes unexpectedly. \\
We recorded the $m/Z$ ratio over a range of \SI{38}{} to \SI{330}{\atomicmassunit \per \elementarycharge}. Fig. \ref{fig_magic_peak} shows the result of the measurement. We can already see that number 13 and 14 do not follow the decreasing trend, but they present a plateau, furthermore at 15 one notices a drop in intensity. To further investigate these magic numbers, we integrated the area beneath each peak and compared these values to each other. This is shown in fig. \ref{fig_magic_simple}.

\begin{figure}[H]
	\centering
	\includegraphics[width = 0.8 \textwidth]{magic_peaks}
	\caption{Depiction of the $m/z$ spectrum of \ch{Ne}-clusters from \SI{38}{} to \SI{330}{\atomicmassunit \per \elementarycharge} in logarithmic scale.}
	\label{fig_magic_peak}
\end{figure}

\begin{figure}[H]
  \centering{}
  \begin{subfigure}[t]{0.45 \textwidth}
    \centering
    \includegraphics[height=6cm]{magic_area.pdf}
    \caption{Area beneath peaks.}
  \end{subfigure}
  ~
  \begin{subfigure}[t]{0.45 \textwidth}
    \centering
    \includegraphics[height=6cm]{magic_height.pdf}
    \caption{Height of Peaks. }
  \end{subfigure}
  \caption{Plot of the area beneath the peaks and the height of the peaks in Fig. \ref{fig_magic_peak}}
  \label{fig_magic_simple}
\end{figure}
As can be seen in fig. \ref{fig_magic_simple} there is a plateau in the size of the clusters at 12, 13 and 14 atom clusters. In the other peaks there seems to be an exponential decline in cluster size. Since the peak at 12 seems to still be part of the declining line, we assume that only 13 and 14 are indeed magic numbers in the Neon cluster spectrum, although they are not really prominent peaks compared to the peaks in the low size clusters. \\
This seems to be in accordance with \cite{paper_scheier}, where it is stated that 13 is a magic number for clusters of icosahedral shape. They too reach the conclusion that \ch{Ne14+} has as a magic number of atoms since it is followed by a comparably steep decline in intensity, which is also visible in fig. \ref{fig_magic_peak}. Even though $14$ is not listed as a magic number for any geometrical shape in \cite{bergmann}, it visibly stands out in fig. \ref{fig_magic_simple}. Since its magic properties are not of geometrical origin, we assume that chemical properties of \ch{Ne} are responsible for $14$ being a magic number.

\subsection{Appearance energy for Ne and Ne$_2$}
In the next section the appearance energy of the ionized monomer and dimer \ch{Ne+} and \ch{Ne2+} is studied. To measure the ionization energy of the Neon atom the mass selector was fixed to \SI{20.6}{\atomicmassunit}, and the energy of the electron beam was varied. The energy spectrum of the monomer is shown in fig. \ref{fig_energy_monomer}. By fitting a constant and a linear function into the recoded data, one is able to estimate the appearance energy.
\begin{figure}[H]
	\centering
	\includegraphics[width = 0.8 \textwidth]{energy_ne.pdf}
	\caption{Plot of the detected signal depending on the electron energy. The displacement of the constant fit is $c=\SI{1.21(4)}{\hertz}$, while the parameters of the linear function are $a = \SI{165(3)}{\hertz \per \electronvolt}$ and $b = \SI{-3400(60)}{\hertz}$. This yields an intersection at \SI{20.53(1)}{\electronvolt}. }
	\label{fig_energy_monomer}
\end{figure} \newpage
The intersection of the constant and the linear function, which corresponds to the appearance energy, is found to be at \SI{20.53(1)}{\electronvolt} in fig. \ref{fig_energy_monomer}. According to \cite{neon_nist} the ionization energy of \ch{Ne} is \SI{21.56454}{\electronvolt}. This result allows us to calibrate the energy axis for the appearance energy of the Neon dimer. \\
The calculation of the appearance energy of the dimer is done in an analogous way to the monomer. The mass selector is set to \SI{40.6}{\atomicmassunit}, then the energy range is again scanned in the range depicted in fig. \ref{fig_energy_dimer}.
\begin{figure}[H]
	\centering
	\includegraphics[width = 0.8 \textwidth]{energy_ne2.pdf}
	\caption{Depiction of the appearance energy for a Neon dimer. The energy axis was calibrated with the result of the monomer. The parameters determined through the fit were $c = \SI{1.18(5)}{\hertz}$ for the constant function, $a = \SI{9(1)}{\hertz \per \electronvolt}$ as the slope and $b = \SI{-200(20)}{\hertz}$ as the offset of the linear function. The intersection is found at \SI{21.32(7)}{\electronvolt}. Note that the intensity scale is different than in fig. \ref{fig_energy_monomer}. }
	\label{fig_energy_dimer}
\end{figure}
The intersection in fig. \ref{fig_energy_dimer} is located at \SI{21.32(7)}{\electronvolt}, which is slightly lower than the previous value, also it has a larger error. From what we see it takes approximately as much energy to ionize a Neon atom as it does to ionize a Neon dimer. However, one will notice that even though fig. \ref{fig_energy_monomer} and \ref{fig_energy_dimer} look very similar the intensity scale is different by about a factor of $15$. \\
We assume this is simply because there is more monomers in the gas that is ionized than there are dimers. Looking at fig. \ref{isotopespectrum}, one can see that the intensity is increasing the smaller the cluster is. Therefore it seems plausible that the monomer is actually the most abundant component of the gas that is ionized.

\subsection{Pick-up}

 In the last part of the experiment we studied the impact of a pick-up gas on the clusters. We did not record the spectrum ourselves, but we were given already existing data. Simple air was used as pick-up. The mass spectrum is shown in fig. \ref{pickup}. One immediately notices two peaks on the left at 28.3 and \SI{32.3}{\atomicmassunit \per \elementarycharge}. These are from the pickup gas, i.e. air which consists mainly of \ch{N2} and \ch{O2} which are the peaks we see. In fig. \ref{pickupzoom} a close-up on the smaller peaks can be seen. Those peaks are one or two orders of magnitude lower than the peaks from air components or pure \ch{Ne} clusters, but they still hold interesting information. First of all we recognize the same isotope peaks of section \ref{section:isotopes}, but this time there are more peaks in between the pure clusters, which are clusters with the pickup gas nitrogen, oxygen, or both. In table \ref{pickupresults} we summarize all peaks, and our best guess of what they could be. We also observe the same systematic error in the $m/z$ ratio as in section \ref{section:isotopes}.
In the results we notice clusters that contain pick-up gas. In two instances the neon cluster picked up the molecule \ch{N2}, in one case the pick-up was one oxygen, and in another case the cluster picked up a \ch{NO} molecule.
 Furthermore, we notice that in this experiment, clusters with a size of more than $4$ neon atoms did not pick up any other element.

\begin{figure}[H]
	\centering
	\includegraphics[width =\textwidth]{pickup}
	\caption{Mass spectrum of Neon clusters with air as pickup.}
	\label{pickup}
\end{figure}
\begin{figure}[H]
	\centering
	\includegraphics[width =\textwidth]{pickupzoom}
	\caption{Close-up of the mass spectrum of Neon clusters with air as pickup.}
	\label{pickupzoom}
\end{figure}

\begin{table}[H]
\centering
\caption{Results for the Neon cluster spectrum with pickup. All theoretical calculations are made with the software \cite{umc}}\label{pickupresults}
\begin{tabular}{ccc} \toprule
Measured peak (u/e) & Theoretical (u/e) & Cluster composition (charge +) \\ \midrule
40.5 & 39.98 & \ch{^{20}Ne2}\\
42.5 & 41.98 & \ch{(^{20}Ne)(^{22}Ne)}\\
44.5 & 44.00 &\ch{N2O}\\
56.5 & 55.98 & \ch{Ne2O}\\ \midrule
60.5 & 59.98 & \ch{^{20}Ne3}\\
62.5 & 61.98 & \ch{(^{20}Ne)2(^{22}Ne)}\\
64.5 & 63.98 & \ch{(^{20}Ne)(^{22}Ne)2}\\
68.5 & 67.99 & \ch{Ne2N2}\\\midrule
80.4& 79.97 & \ch{^{20}Ne4}\\
82.4& 81.97 & \ch{(^{20}Ne)3(^{22}Ne)}\\
84.4 & 83.97 & \ch{(^{20}Ne)2(^{22}Ne)2}\\
88.5 & 87.98 & \ch{N2Ne3}\\
90.5 & 89.97 & \ch{Ne3NO}\\\midrule
100.5 & 99.96 & \ch{^{20}Ne5}\\
102.5 & 101.96 & \ch{(^{20}Ne)4(^{22}Ne)}\\
104.7 & 103.96 & \ch{(^{20}Ne)3(^{22}Ne)2}\\\midrule
120.6 & 119.95 & \ch{^{20}Ne6}\\
122.6 & 121.95 & \ch{(^{20}Ne)5(^{22}Ne)}\\
124.6 & 123.95 & \ch{(^{20}Ne)4(^{22}Ne)2}\\
\bottomrule
\end{tabular}
\end{table}

\section{Conclusion}
In the course of this experiment we measured the isotope abundance in Neon clusters. We found that \ch{^{22}Ne} is slightly more abundant than expected, although we are not sure if this is due to a shifted abundance in the source or because of chemical properties. Further we looked at magic numbers in the spectrum and found such numbers at $13$ and $14$, which is in agreement with \cite{paper_scheier}. Next, we measured an energy spectrum for the ionization of \ch{Ne} and used the result to calibrate the energy scale. This calibration was then used to determine the ionization energy of \ch{Ne2}, which was found to be \SI{21.32(7)}{\electronvolt}. Finally, a pick-up measurement was analyzed where we found that mostly \ch{N2} and \ch{NO} was picked up by the \ch{Ne} clusters.

\begin{thebibliography}{99}

\bibitem{script}
\textsc{Stephan Denifl}, \textit{FP3‐Praktikumsversuch – Edelgascluster/Rare gas clusters (SS 2018)}, 2018

\bibitem{bergmann}
\textsc{Bergmann-Schäfer}, \textit{Gase, Nanosysteme, Flüssigkeiten}, \textit{Band 5} (de Gruyter, 2005)

\bibitem{ms_book}
\textsc{Jürgen H. Gross}, \textit{Mass Spectrometry}, Springer, 2nd Edition, 2011

\bibitem{neon_nist}
\textsc{Nist}, \textit{Basic Atomic Spectroscopic Data - Neon (Ne)}, \url{https://www.physics.nist.gov/PhysRefData/Handbook/Tables/neontable1.htm}

\bibitem{paper_scheier}
\textsc{T.D. Märk, P. Scheier}, \textit{PRODUCTION AND STABILITY OF NEON CLUSTER IONS UP TO } \ch{Ne90+}, Chemical Physics Letters Vol. 137, Number 3, 12 June 1987

\bibitem{umc}
\textsc{Matthias Letzel}, \textit{Universal Mass Calculator - Student Edition}, Universität Münster, \url{https://www.uni-muenster.de/Chemie.oc/ms/downloads.html}
\end{thebibliography}

\end{document}
