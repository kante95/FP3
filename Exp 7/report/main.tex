\documentclass[a4paper,10pt]{article}

\usepackage[utf8]{inputenc}
%\usepackage[T1]{fontenc}

\usepackage{textcomp}           % Extra Symbole (Grad Celsius etc.)
\usepackage{amssymb,amsmath}    % Schöne Formeln (AMS = American Mathematical Society)
\usepackage{graphicx}           % Bilder und Seitenränder
\usepackage{subcaption}			% captions for subfigures
\usepackage{booktabs}           % Schönere Tabellen
\usepackage{colortbl}           % Farbige Tabellen

%\usepackage{tcolorbox}			% schöne bunte Boxen
\usepackage{mathtools}			% \mathclap für ordentliche \underbrace-			environments
\usepackage[left=2cm,right=2cm,top=2cm,bottom=2cm]{geometry}			% Pagelayout mit \newgeometry, \restoregeometry
\usepackage{float}
\usepackage{wrapfig}
\usepackage{enumitem}
\usepackage{float}
\usepackage{braket}
\usepackage{caption}
\usepackage[per-mode=fraction,output-decimal-marker={.},binary-units=true,separate-uncertainty=true]{siunitx}
\usepackage[breaklinks=true,colorlinks=true,linkcolor=blue,urlcolor=blue,citecolor=blue]{hyperref}
\usepackage{physics}
\usepackage{url}
\usepackage{subcaption}
\usepackage{calrsfs}
\DeclareMathAlphabet{\pazocal}{OMS}{zplm}{m}{n}
\usepackage{tikz}
\usetikzlibrary{decorations, positioning, intersections, calc, shapes,arrows, scopes}
\usepackage{pgfplots}
\usepackage{bodegraph}
\usepackage{circuitikz}
\usepackage{chemfig}
\usepackage{chemformula}
\usepackage[toc,page]{appendix}
\graphicspath{{./img/}}
\usepackage{verbatim}

\DeclareSIUnit\elementarycharge{e}

\newcommand{\dif}{\mathrm{d}}

\bibliographystyle{unsrtnat}

\renewcommand{\k}{\mathbf{k}}
\begin{document}
\begin{titlepage}
 \begin{center}
	\Large{Advanced laboratory course 3}
	\end{center}
	\begin{center}
	 \LARGE{\textbf{FP3 - Rare gas clusters}}
	\end{center}

	\begin{center}

	\large Marco \textsc{Canteri} \\
	marco.canteri@student.uibk.ac.at\\
	\large Maximilian Gerold \textsc{Münst} \\
	maximilian.muenst@student.uibk.ac.at
	\end{center}

	\begin{center}
	\vspace{1cm}
	Innsbruck, \today
	\vspace{1cm}
	\end{center}

	\begin{abstract}
	We created neon clusters via superbeam expansion technique, we then studied the isotope population of the clusters, magic numbers, and energy appearance. Moreover, we studied the 
	impact of air as pick-up gas in the neon clusters.
    \end{abstract}
    \vspace{1cm}

	\begin{center}
	\includegraphics[scale=0.56]{img/uibk}
	\end{center}

\end{titlepage}


\section{Introduction}
In this experiment we explored the world of clusters, which are an aggregate of atoms with no particular structure. Cluster are the missing link between molecules and condensed matter, their wide applications are fundamental in nanotechnology, where clusters are the building blocks of the technology.
\section{Theoretical Background}
What's neon? And what's the meaning of a neon cluster? Can we reach god with it? Can we finally understand the meaning of life?\\
These and more questions will be answered in this section
\section{Experimental Setup}
The cluster source is a jet noozle. Neon gas is stored in a stagnation chamber at high pressure and high temperature. It is then released in a vacuum chamber where the gas adiabatically expand at supersonic speeds. A skimmer collect the central part of the gas jet to guide it into the cluster chamber, where the clusters are formed by nucleation. The formed clusters are subsequently collided with an electron beam, such that the neutral clusters are positively ionized via electron ionization. The ions now can be analyzed with a mass spectrometer, for this experiment we used a linear quadrupole mass spectrometer.\\
The electron beam is created by evaporation from a filament, the beam is focused with a set of electron lenses and filtered by energy with an hemispherical electron monochromator.

\section{Analysis}
\subsection{Isotope population}
\subsection{Magic numbers}
\subsection{Appearance energy for Ne and Ne$_2$}
\subsection{Pick-up}
\section{Conclusion}

\begin{thebibliography}{99}

\end{thebibliography}

\end{document}
